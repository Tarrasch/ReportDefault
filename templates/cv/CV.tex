\documentclass[10pt,twoside,a4paper,aps,twocolumn,preprintnumbers,nofootinbib]{revtex4}
\usepackage[utf8]{inputenc}
\usepackage[english]{babel}
\usepackage{verbatim}
\usepackage{textcomp}
\usepackage{amsfonts}
\usepackage{url}
\usepackage{bbold}
\usepackage{latexsym}
\usepackage{lmodern} %Neat font
\usepackage{dcolumn}
\usepackage[pdftex]{graphicx}
\usepackage[pdftex,bookmarks,colorlinks]{hyperref}
\usepackage{booktabs}
\usepackage{blindtext}
\usepackage[includeheadfoot,margin=2cm]{geometry}
\usepackage{soul}

\documentclass[10pt,twoside,a4paper,aps,twocolumn,preprintnumbers,nofootinbib]{revtex4}
\usepackage[utf8]{inputenc}
\usepackage[english]{babel}
\usepackage{verbatim}
\usepackage{textcomp}
\usepackage{amsfonts}
\usepackage{url}
\usepackage{bbold}
\usepackage{latexsym}
\usepackage{lmodern} %Neat font
\usepackage{dcolumn}
\usepackage[pdftex]{graphicx}
\usepackage[pdftex,bookmarks,colorlinks]{hyperref}
\usepackage{booktabs}
\usepackage{blindtext}
\usepackage[includeheadfoot,margin=2cm]{geometry}
\usepackage{soul}

\newcommand{\bra}[1]{\langle #1|}
\newcommand{\ket}[1]{|#1\rangle}
\newcommand{\braket}[2]{\langle #1|#2\rangle}
\newcommand{\bracket}[3]{\langle #1|#2|#3\rangle}
\newcommand{\expect}[1]{\langle #1 \rangle}
\newcommand{\assumption}[1]{

{\bf Assumption:} #1}
\newcommand{\paren}[1]{\ensuremath{\left(#1\right)}}
\newcommand{\sparen}[1]{\ensuremath{\left[#1\right]}}
\newcommand{\cparen}[1]{\ensuremath{\left\{#1\right\}}}
\newcommand{\abs}[1]{\ensuremath{\left|#1\right|}}


\newcommand{\angstrom}{\textup{\AA}}

\newcommand{\nuc}[3][ ]{\ensuremath{^{#2}_{#1}{\mathrm{#3}}}}
\newcommand{\atom}[1]{\ensuremath{\mathrm{#1}}}
\newcommand{\tskip}[1]{\vphantom{\rule{0 mm}{#1 em}}}

\newcommand{\dd}{\ensuremath{\mathrm{d}}}


\newcommand*\circled[1]{\tikz[baseline=(char.base)]{
            \node[shape=circle,draw,inner sep=2pt] (char) {#1};}}

\newcommand{\panicmargindecrement}[2]{
\addtolength{\oddsidemargin}{-#1}
\addtolength{\evensidemargin}{-#1}
\addtolength{\textwidth}{#1*\real{2}}
\addtolength{\topmargin}{-#2}
\addtolength{\textheight}{#2*\real{2}}
}

\newcommand{\U}[2][]{\ensuremath{\unit[#1]{#2}}}


%Typesetting the Ricci tensor in a neat way:
\newlength{\dotlength}
\settowidth{\dotlength}{$\cdot$}
\newlength{\argX}
\newcommand{\centereddot}[1]{ \settowidth{\argX}{#1} \setlength{\argX}{0.4\argX-0.5\dotlength} \hspace{\argX}\ensuremath{\cdot}\hspace{\argX}}
\newcommand{\Ruuuu}[4]{\ensuremath{R^{#1 #2 #3 #4}}}
\newcommand{\Ruuud}[4]{\ensuremath{R^{#1 #2 #3}_{\centereddot{#1}\centereddot{#2}\centereddot{#3} #4}}}
\newcommand{\Ruudu}[4]{\ensuremath{R^{#1 #2 \centereddot{#3} #4}_{\centereddot{#1} \centereddot{#2} #3 }}}
\newcommand{\Ruudd}[4]{\ensuremath{R^{#1 #2}_{\centereddot{#1} \centereddot{#2} #3 #4}}}
\newcommand{\Ruduu}[4]{\ensuremath{R^{#1 \centereddot{#2} #3 #4}_{\centereddot{#1} #2 }}}
\newcommand{\Rudud}[4]{\ensuremath{R^{#1 \centereddot{#2} #3}_{\centereddot{#1} #2 \centereddot{#3} #4 }}}
\newcommand{\Ruddu}[4]{\ensuremath{R^{#1 \centereddot{#2}\centereddot{#3} #4}_{\centereddot{#1} #2 #3 }}}
\newcommand{\Ruddd}[4]{\ensuremath{R^{#1}_{\centereddot{#1} #2 #3 #4}}}

\newcommand{\Rduuu}[4]{\ensuremath{R^{\centereddot{#1} #2 #3 #4}_{#1}}}
\newcommand{\Rduud}[4]{\ensuremath{R^{\centereddot{#1} #2 #3}_{#1 \centereddot{#2} \centereddot{#3} #4}}}
\newcommand{\Rdudu}[4]{\ensuremath{R^{\centereddot{#1} #2 \centereddot{#3} #4}_{#1 \centereddot{#2} #3}}}
\newcommand{\Rdudd}[4]{\ensuremath{R^{\centereddot{#1} #2}_{#1 \centereddot{#2} #3 #4}}}
\newcommand{\Rdduu}[4]{\ensuremath{R^{\centereddot{#1}\centereddot{#2} #3 #4}_{#1 #2}}}
\newcommand{\Rddud}[4]{\ensuremath{R^{\centereddot{#1}\centereddot{#2} #3}_{#1 #2 \centereddot{#3} #4}}}
\newcommand{\Rdddu}[4]{\ensuremath{R^{\centereddot{#1}\centereddot{#2}\centereddot{#3} #4}_{#1 #2 #3 }}}
\newcommand{\Rdddd}[4]{\ensuremath{R_{#1 #2 #3 #4}}}

\panicmargindecrement{-0.9cm}{0.5cm}
\addtolength{\textheight}{1.5cm}
% Change the width of the text, and use the horizontal offset to center text.
%\addtolength{\textwidth}{0.2in}
%\addtolength{\hoffset}{-0.1in}
\setlength{\parskip}{0.3cm}
\pagestyle{fancy}
\fancyhf{} 
\lhead{\emph{\AuthorName}}
\rhead{\scriptsize{\thepage\ (\pageref{LastPage}})} 
\renewcommand\headrulewidth{0pt} % Removes funny header line 
\thispagestyle{empty}

\sectionfont{\rmfamily\mdseries\Large}
\subsectionfont{\rmfamily\mdseries\itshape\large}



% New definition of square root:
% it renames \sqrt as \oldsqrt
\let\oldsqrt\sqrt
% it defines the new \sqrt in terms of the old one
\def\sqrt{\mathpalette\DHLhksqrt}
\def\DHLhksqrt#1#2{%
\setbox0=\hbox{$#1\oldsqrt{#2\,}$}\dimen0=\ht0
\advance\dimen0-0.2\ht0
\setbox2=\hbox{\vrule height\ht0 depth -\dimen0}%
{\box0\lower0.4pt\box2}}
%\numberwithin{equation}{section}
\setlength{\columnsep}{0.8cm}
%\setlength{\textwidth}{15cm}

% Fancyheader shortcuts

\renewcommand{\topfraction}{0.85}
\renewcommand{\textfraction}{0.1}
\renewcommand{\floatpagefraction}{0.85}

\graphicspath{{fig/}}

% PDF Metadata and link styles
\def\footerlink{\AuthorURL}
\hypersetup{
  colorlinks=true,
  urlcolor=black,
  citecolor=black,
  filecolor=black,
  linkcolor=black,
  pdfauthor={\AuthorName},
  pdfkeywords={Cover Letter, \AuthorName},
  pdftitle={\AuthorName: Cover Letter},
  pdfsubject={Cover Letter},
  pdfcreator=LaTeX,
  pdfproducer=\AuthorName,
  pdfpagemode=UseNone
}

% Make lists without bullets
\renewenvironment{itemize}{
  \begin{list}{}{
    \setlength{\leftmargin}{1.5em}
  }
}{
  \end{list}
}

% Don't indent paragraphs.
\setlength\parindent{0em}

% Change the width of the text, and use the horizontal offset to center text.
%\addtolength{\textwidth}{0.2in}
%\addtolength{\hoffset}{-0.1in}
\setlength{\parskip}{0.3cm}
\pagestyle{fancy}
\fancyhf{} 
\lhead{\emph{\AuthorName}}
\rhead{\scriptsize{\thepage\ (\pageref{LastPage}})} 
\renewcommand\headrulewidth{0pt} % Removes funny header line 
\thispagestyle{empty}

\sectionfont{\rmfamily\mdseries\Large}
\subsectionfont{\rmfamily\mdseries\itshape\large}



% New definition of square root:
% it renames \sqrt as \oldsqrt
\let\oldsqrt\sqrt
% it defines the new \sqrt in terms of the old one
\def\sqrt{\mathpalette\DHLhksqrt}
\def\DHLhksqrt#1#2{%
\setbox0=\hbox{$#1\oldsqrt{#2\,}$}\dimen0=\ht0
\advance\dimen0-0.2\ht0
\setbox2=\hbox{\vrule height\ht0 depth -\dimen0}%
{\box0\lower0.4pt\box2}}
%\numberwithin{equation}{section}
\setlength{\columnsep}{0.8cm}
%\setlength{\textwidth}{15cm}

% Fancyheader shortcuts

\renewcommand{\topfraction}{0.85}
\renewcommand{\textfraction}{0.1}
\renewcommand{\floatpagefraction}{0.85}

\graphicspath{{fig/}}

% PDF Metadata and link styles
\def\footerlink{\AuthorURL}
\hypersetup{
  colorlinks=true,
  urlcolor=black,
  citecolor=black,
  filecolor=black,
  linkcolor=black,
  pdfauthor={\AuthorName},
  pdfkeywords={Cover Letter, \AuthorName},
  pdftitle={\AuthorName: Cover Letter},
  pdfsubject={Cover Letter},
  pdfcreator=LaTeX,
  pdfproducer=\AuthorName,
  pdfpagemode=UseNone
}

% Make lists without bullets
\renewenvironment{itemize}{
  \begin{list}{}{
    \setlength{\leftmargin}{1.5em}
  }
}{
  \end{list}
}

% Don't indent paragraphs.
\setlength\parindent{0em}





\begin{document}
\vspace*{-0.7in}
\begin{minipage}{0.45\linewidth}
{\Huge Curriculum Vitae }
\end{minipage}
\begin{minipage}{0.45\linewidth}
{\hfill \small \ddmmyyyydate \today}
\end{minipage}
\vspace*{0.12in}
{\tiny }\\
{\huge \name}


%\maketitle
\vspace{0.1in}
\begin{minipage}{0.45\linewidth}
  27 Prince George's Park Residence 2\\
  Block 8, Level 2, Room E \\
  Singapore, Singapore 118425 \\
    
\end{minipage}
\begin{minipage}{0.45\linewidth}
  \begin{tabular}{ll}
    Phone: & +6581249770\\
    Email: & \href{mailto:rikard.lundmark@mensa.se}{rikard.lundmark@mensa.se}\\
    Date of birth: & July 16, 1990 \\
    Nationality: & Swedish
  \end{tabular}
\end{minipage}

\section*{Education}

\subsection*{Ongoing}
\begin{description}
\item[Chalmers Tekniska Högskola (2009-2014)] Civilingenjör Teknisk Fysik (Master of Science in Engineering Physics). Up to the date of this document I have highest grades in all subjects.
\item[National University of Singapore (2012-2013)] Exchange student. Awarded the \emph{ASEM-DUO} scholarship.
\end{description}

\subsection*{Finished}
\begin{description}
\item[Primary school (1997-2006)] Åsenskolan, Anderstorp. Highest grades in all theoretical subjects.
\item[Secondary school (2006-2009)] The Natural Science programme at Gislaveds Gymnasium, Gislaved. Course of study: extended. Highest grades in all subjects. Finished all mathematics-, physics-, and chemistry courses that the school offered in less than 2 years instead of the normal 3 years, and was awarded scholarships and premiums for my outstanding results.
\item[University education] University courses studied simultaneously with my secondary school education, corresponding to 30 ECTS credits distributed on four 7,5 credits courses, all with the highest grade: \emph{Högskolan i Gävle (2008)} - Linear Algebra, Advanced Linear Algebra, Single Variable Calculus. \emph{Umeå Universitet (2008-2009)} -  Analogue Electronics I.
\end{description}

\subsection*{Research}
During the {\bf summer 2011} I spent five weeks doing a research project at the Subatomic physics research group (Fundamental Physics department) at Chalmers University of Technology. The aim of the project was to evaluate a proposed particle-detector design using mathematical models and self-written simulations. During the {\bf spring 2012} the project was finished and became my bachelor's thesis (\url{http://fy.chalmers.se/subatom/kand/2012/precalib/}).

\section*{Employment}

\begin{description}
\item[2010]\emph{Chalmers University of Technology}, Göteborg. Teaching assistant, Computer Science and Engineering. I mainly answered questions from students and graded their assignnments in the computer introduction course. I worked about 5 hours per week during the course (which is 7 weeks). 
\item[2011]\emph{Chalmers University of Technology}, Göteborg. Teaching assistant, Fundamental Physics. I mainly answered questions from students and graded their assignnments in the computer introduction course. I worked about 5 hours per week during the course (which is 7 weeks). 
\item[2011-2012]\emph{Chalmers University of Technology}, Göteborg. Teaching assistant, Mathematical Sciences. I held classes for about 30 students a time, a few times a week. The extent of the work was 12-15 hours per week.
\end{description}


\subsection*{Summer jobs}
\begin{description}
\item[2005-2008] \emph{Wibe Defem AB}, Anderstorp. Industrial work, my tasks were e.g. to pack items for delivery.
\item[2009] \emph{Recticel AB}, Gislaved. Industrial work, my tasks were e.g. to pack mattresses for delivery to customers. 
\item[2010-2012] \emph{Unfors AB}, Göteborg. Programmer, R\&D. I developed software for different purposes in \verb+C\#+, \verb+ASP.NET+ and \verb.C++.. 
\end{description}


\subsection*{Miscellaneous}
During {\bf 2009} I spent two weeks of work practice at \emph{Blue Tree} in Galway, Ireland, developing a web interface for retrieving statistics from databases and for server monitoring. 

\section*{Extracurricular activities}
\begin{description}
\item[Anderstorps Tennisklubb 2004-2009] \emph{Webmaster}. I was responsible for administrating and updating Anderstorps tennisclub's web page.
\item[Projektbanken 2009-2010] \emph{Project group member}. \href{http://www.fuf.org/projektbanken/}{Projektbanken} is a cooperation project between \href{http://www.fuf.org/}{Swedish Feder-} \href{http://www.fuf.org/}{ation of Young Scientists (FUF)} and Swedish universities providing direct contact between researchers and secondary school students. This makes it possible for the most ambitious students to do research projects before starting university, and thus aims to stimulate their scientific interest. As one of six members in the project group, my main responsibilities were IT and contact with researchers and teachers.
\item[F-spexet 2011] \emph{Organizing member}. \href{http:///www.f-spexet.se}{F-spexet} is a student organisation for physics students at Chalmers University of Technology, whose purpose is to create a \emph{spex} (student comedy show) each year.
\item[Intize 2011-2012] \emph{Mentor}. \href{http://www.intize.org}{Intize} is an organization for helping students learning mathematics by assigning them university students as mentors to coach them in their studies. As a mentor I had four students, which I helped for a few hours every week. 
\end{description}

\section*{Other qualifications}

\subsection*{Contests}
During my primary and secondary school education I have participated in several national and international contests and achieved outstanding results, which in several cases qualified me to represent Sweden in international competitions. A selection of these are [\emph{Name of competition (year, national placement, country of international competition if applicable, comments)}]: 
\uline{European Union Science Olympiad} (2006, top 6, Belgium), \hspace{0.07cm}
\uline{Hogstadiets Matematiktavling} (2006, 8, N/A, national competition), \hspace{0.07cm}
\uline{Skolornas Matematiktävling} (2008, 8, N/A, national competition), \hspace{0.07cm}
\uline{Nordic Mathematical Contest} (2009, 25, N/A), \hspace{0.07cm}
\uline{International Chemistry Olympiad} (2008, top $\approx 20$, N/A), \hspace{0.07cm}
\uline{International Physics Olympiad} (2008, 5, Vietnam), \hspace{0.07cm}
\uline{Baltic Olympiad in Informatics} (2009, 4, Sweden), \hspace{0.07cm}
\uline{International Olympiad in Informatics} (2009, 4, Bulgaria), \hspace{0.07cm}
\uline{Utställningen Unga Forskare} (2009, final round), \hspace{0.07cm}
\uline{International Chemistry Olympiad} (2009, 4, Great Britain, did not attend international the competition due to participation in the Physics Olympiad), \hspace{0.07cm}
\uline{International Physics Olympiad} (2009, 1, Mexico, won honourable mention on international level), \hspace{0.07cm}
\uline{International Biology Olympiad} (2009, 3, Japan, did not attend the international competition due to participation in the Physics Olympiad), \hspace{0.07cm}
\uline{Teknik-SM} (2010, top 6, N/A, national competition), \hspace{0.07cm}
\uline{Nordic Mathematical Contest} (2011, 5, N/A, team competition), \hspace{0.07cm}
\uline{Teknik-SM} (2012, top 8, N/A, national competition)

\section*{Miscellaneous}
\subsection*{Computer proficiency}
\begin{description}
\item[Operating systems] Linux, Windows
\item[Programming languages] Java, C++, C\#, \verb+MATLAB+, Visual Basic, ASP, ASP.NET, HTML, \LaTeX, PHP, \verb+LabVIEW+, LUA. I learn new systems and programming languages in a matter of days.
\end{description}
\vspace{-0.1in}

\subsection*{Language proficiency}
\textbf{Swedish} (native), \textbf{English} (excellent), \textbf{German} (good), \textbf{Mandarin} (very basic).

\subsection*{Intelligence Quotient}
According to a normed IQ test, \emph{Figure Reasoning Test A}, I have an intelligence quotient of 135 or higher on the Wechsler IQ scale.

\section*{Referees}
Given upon request.

\end{document}